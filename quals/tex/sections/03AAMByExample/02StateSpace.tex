%%%%%%%%%%%%%%%%%%%%%%%
%% Cutting Recursion %%
%%%%%%%%%%%%%%%%%%%%%%%

To arrive at a computable analysis we must abstract the infinite concrete state
space by a finite abstract state space.
%
In the AAM methodology, we first identify recursion in the state space and
replace it with indirection through addresses and a store.
%
\begin{figure}[H]
\haskell{sections/03AAMByExample/hs/02AbstractSSCut.hs}
\caption*{Cutting Recursion}
\end{figure} 
\noindent
%
Abstract time is also introduced into the state space as a constantly
increasing value from which to allocate fresh addresses.
%
As a result of this change in the state space, the semantics will need to
change accordingly.
%
Variable lookup must fetch an address from the environment, and then fetch the
corresponding value bound to that address from the store.
%
Variable binding must allocate globally fresh addresses from the current time,
binding the formal parameter to the address in the environment and the address
to the value in the store.

%%%%%%%%%%%%%%%%%%%%%%%%
%% Abstract Addresses %%
%%%%%%%%%%%%%%%%%%%%%%%%

Now that recursion is eliminated from the state space we take aim at unbounded
parts of the state space, with the addresses and time that we just introduced
being the first to go.
%
Rather than make a single choice of abstract address and time, we leave this
choice as a parameter.
%
All that is required is:
\begin{itemize}
\item a time \h|tzero| to begin execution
\item a function \h|tick| which moves time forward
\item a function \h|alloc| which allocates an address for binding a formal
      parameter at the current time
\end{itemize}
%
We encode this interface using a Haskell type class \h|AAM aam| with associated
types \h|Addr aam| and \h|Time aam|.
%
\begin{figure}[H]
\haskell{sections/03AAMByExample/hs/02AbstractSSAAM.hs}
\caption*{Address and Time Abstraction}
\end{figure}
\noindent
%
An implementation of \h|AAM aam0| for a particular type \p|aam0| must pick two
types, \h|Addr aam0| and \h|Time aam0|, and then implement \h|tzero|, \h|tick|,
\h|alloc|, for those types.
%
The type |aam0| its self will only used as a type level token for selecting
associated types at the type level, and will have a single inhabitant for
selecting type class functions at the value level.
%
(If Haskell had a proper module system this would all be better expressed as a
module interface containing a type and functions over that type, rather than an
associated type, type class, and proxy singleton type.)

%%%%%%%%%%%%%%%%%%%%%%%%%%%%%%%%%
%% Introducting Nondeterminism %%
%%%%%%%%%%%%%%%%%%%%%%%%%%%%%%%%%

Following the AAM methodology, we realize that in a world with finite addresses
but potentially infinite executions, there is a chance we will need to reuse an
address a potentially unbounded number of times.
%
This causes us to introduce a powerset both around the state space and into the
domain of the heap, as a single address may now point to multiple values, and
variable reference may return multiple values.
%
\begin{figure}[H]
\haskell{sections/03AAMByExample/hs/02AbstractSSNonDet.hs}
\caption*{Introducing Nondeterminism}
\end{figure}
\noindent
%
As a result of this abstraction, the semantics must consider multiple control
paths where before there was only one.

%%%%%%%%%%%%%%%%%%%%%%%
%% Delta Abstraction %%
%%%%%%%%%%%%%%%%%%%%%%%

Now we have a \textit{nondterministic} semantics paramterized by abstract
addresses--but we are not yet finished.
%
The state space is still infinite due to the unboundedness of integer literal
values.
%
(Even if we took the semantics of finite machine integers, $2^{32}$ is still much
too large a state space to be exploring in a practical analysis.)
%
To curb this we introduce yet another axis of parameterization with the intent
of recovering multiple analyses from the choice later.
%
In the Haskell implementation we represent this axis with the type class
\h|Delta delta| and associated type \h|Val delta|.
%
\begin{figure}[H]
\haskell{sections/03AAMByExample/hs/02AbstractSSDelta.hs}
\caption*{(Final) Delta Abstraction}
\end{figure}
\noindent
%
The reason for replacing the algebraic data type \h|Val| with type class
\h|Delta delta| is that different analyses may want to make different choices
in their value representations.
%
All that matters is that we have \textit{introduction} forms for literals and
closures, a \textit{denotation} function for primitive operations, and
\textit{elimination} forms for booleans and closures.
%
Eliminators are only required for booleans and closures because those are the
only values scrutenized in the control flow of the semantics.
