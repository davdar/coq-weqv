Now that we have designed an abstract state space--albeit highly
parameterized--we must turn to implementing its semantics.
%
The goal in this entire exercise is to develop the abstract state space and
semantics in such a way that concrete and abstract interpreters can be derived
from a single implementation.

%--%

Following the structure of the concrete semantics, we adapt each function to
the new abstract state space.
%
\h|op| is now entirely implemented by the user of the semantics as part of the
\h|Delta| interface.
%
\h|atom| must follow another level of indirection in variable lookups, and must
account for nondeterminism.
%
\haskell{sections/03AAMByExample/hs/03AbstractAtom.hs}
%
\h|call| must use the eliminators provided by \h|Delta delta| to guide control
flow.
%
\haskell{sections/03AAMByExample/hs/03AbstractCall.hs}
%
\h|bindMany| must join stores when allocating addresses, in case the address is
already in use.
%
\haskell{sections/03AAMByExample/hs/03AbstractBindMany.hs}
%
Finally, \h|step| must advance time as it advances each \h|Call| state, and
\h|exec| is modified to a collecting semantics.
%
\haskell{sections/03AAMByExample/hs/03AbstractStep.hs}
%
The collecting semantics keep track of all previously seen states.
%
This means the state being iterated is purely monotonic, and therefore the
iteration must terminate if given a finite instantiation of \p|delta| and
\p|aam|.
